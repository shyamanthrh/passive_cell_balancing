\chapter{Conclusion and Future Scope}

\section{Conclusion}
\indent\indent A cell balancing module constructed is capable of performing passive cell balancing for battery pack containing up to 16 Li-ion cells.
The project tries to reduce the cost of developing a fully functional cell balancing circuit by implementing passive cell balancing rather than active cell balancing.

The Matlab\textsuperscript{\textregistered}-Simulink\textsuperscript{\textregistered} simulation used to test the functionality of the balancing algorithm was run for 15 hours starting with all the cells in different \acrshort{soc} levels. After six charge-discharge cycles, the \acrshort{soc} levels of all the cells were equalized and the full capacity of the battery pack was utilized. This shows that without cell balancing, the nominal capacity of the battery pack is limited to the weakest cell in the battery pack. The balancing time can be decreased by increasing the balancing current. This can be done by reducing the shunt resistor value and choosing the appropriate \acrshort{mosfet}.    



\section{Future Scope}
The project has a wide range of future uses and applications in Battery management systems of electric vehicles, power backup units etc. And the design is simple, cost effective can be integrated to battery management system at ease.

\begin{itemize}
\item The MATLAB\textsuperscript{\textregistered} Simulink\textsuperscript{\textregistered} model can be optimised to make it more efficient and perform balancing at less time without harming the integrity of the system. Thermal modelling of the entire system can be done using Simscape\textsuperscript{\textregistered} to analyse heat radiation and design a cooling solution.
\item The cell balancing module designed can be directly integrated with other management systems to come up with an complete battery monitoring solution for Li-ion battery packs. 
\item The cell balancing modules can be stacked to accommodate more then 16 Li-ion cells using Daisy-chain communication architecture.
\end{itemize}

\section{Learning Outcomes of the Project}
\begin{itemize}
\item Application of knowledge of embedded system design to design balancing module as per specifications.
\item Using technology to ease the engineering process without sacrificing quality of the work
\item Appreciating the viewpoints of different team members and collaborating to find
optimal solution.
\item Learning to use effective communication while interacting with the mentors and
improving presentation skills.
\item Presenting the problem statement and objectives of the project clearly and ensuring
design and implementation of the project meet the specification.
\item Documenting the project as per format expected clearly presenting ideas behind the
design and implementation of each module
\end{itemize}

