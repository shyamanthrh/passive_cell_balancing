
\addcontentsline{toc}{chapter}{Abstract}\vspace{-1cm}
%Border
\begin{tikzpicture}[remember picture, overlay]
  \draw[line width = 4pt] ($(current page.north west) + (0.75in,-0.75in)$) rectangle ($(current page.south east) + (-0.75in,0.75in)$);
\end{tikzpicture}


Growing concerns on climate change and global warming, world as already started looking for eco-friendly alternatives in every domain possible and most of these alternatives use electricity to power up the system which adds in batteries and by virtue batteries are costly needs to be properly maintained. A weak cell in the battery pack will charge and discharge faster than  higher capacity cells and thus it becomes the limiting factor in the run-time of a system. Cell balancing circuits are very necessary in every \acrfull{bms} to equalize cell \acrfull{soc}  so that pack delivers its rated capacity. Passive balancing technique is a simple and low-cost method to equalize the \acrshort{soc} of cells in a battery pack. 

The objective of this project is to develop a passive cell balancing module using cell monitoring \acrshort{ic} BQ76PL455A-Q1 and STM32F103C8T6 microcontroller. A MATLAB\textsuperscript{\textregistered} Simulink\textsuperscript{\textregistered} model of battery pack is designed to aid development of an efficient cell balancing algorithm and to validate its working. Stateflow\textsuperscript{\textregistered}, a Simulink\textsuperscript{\textregistered} add-on is used to develop logic and visualize the algorithm on the run.

A passive cell balancing module based on  BQ76PL455A-Q1 \acrshort{ic} is designed using KiCAD, an open-source PCB design software. STM32F103C8T6 microcontroller was used to run the algorithm, sample cell voltages and perform balancing for a battery pack consisting of 6 cells in series. The module is capable of supporting upto 16 cells in series and can be integrated with Battery management system of Li-ion battery pack with ease. The simulation which was made to validate the balancing algorithm was run for 15 hours and after 6 charge-discharge cycles an imbalanced battery pack got balanced by the application of balancing algorithm. 
 
\pagebreak 